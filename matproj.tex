\documentclass[11pt,a4paper,oneside]{report}
\begin{document}
\title{SymPy, A Python Library for Symbolic Mathematics}
\author{Jason Hsu, Keyan Kousha, Damian Rooney}
\date{Fall 2011}
\maketitle
\section*{Introduction}
Manipulating mathematical equations in symbolic form is rarely a built-in general programming language (GPL) feature. It's certainly not in today's popular programming languages. For mathematicians, this means learning a new domain specific language (DSL) to conduct such work, often barring its functionality from use in a more general context. Where integration between a DSL and a GPL is possible, it is often complicated, or ugly at best; in addition, studying or modifying the code base in these frequently proprietary languages isn't allowed; and features normally found in a GPL are an afterthought and require learning additional new syntax. Beyond the inconvenience, most popular languages in the symbolic math domain also have prohibitive costs (e.g. MatLab, Mathematica, etc.). It would be nice then to have symbolic math be a library to a GPL, written in a GPL, licenced like a GPL, with an implementation as transparent as a GPL's implementation. This is where SymPy, an open source Python library for symbolic math, enters.
\section*{Features}
SymPy is both written and interfaced with entirely in Python (although, according to their webpage, in the future they will supply a C-written core as an option). It's feature list reads as you would expect a computer algebra system's (CAS) feature list to read. It provides things as basic as algebraic simplification and expansion, complex numbers, differentiation, integration, algebraic and differential equation solvers, etc. It also contains modules for algebraic geometry, arbitrary precision floating point arithmetic, statistics, matrices, and quantum physics, among others. Printing, and plotting in both two and three dimensions, is supported as well. SymPy furnishes the user with all the standard fare of a paid CAS, but for free in an well-known, non-proprietary language for easy extensibility and modification.\\\\
 SymPy's features have even found their way into a more elaborate open source CAS, Sage, which is similarly interfaced with via Python (however, unlike SymPy, Sage has components written in several other languages). According to Sage's Wikipedia entry, SymPy's library is used in it's implementation of calculus tools.
 \section*{Innards}
 \subsection*{\small High Level}
\end{document}